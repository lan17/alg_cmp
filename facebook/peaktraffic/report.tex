\documentclass[11pt]{article}

\usepackage{epsf,amsfonts,amssymb,hyperref}

\begin{document}
\title{Solution to "Peak Traffic" Facebook puzzle\cite{peaktraffic}}
\author{Lev A Neiman}

\maketitle

\section{Problem Statement}

Given a directed graph, enumerate all maximal cliques.  Clique is defined as group of nodes in graph that have edges to every other node in clique.  Maximal clique is a clique that is impossible to augment by adding an additional node.

In the base case of 2 nodes, both of them need to have directed edge to each other to be considered a clique.

\section{Solution}

My solution can be explained as a four step process.

\begin{enumerate}
\item Read input file in.  Enumerate every node with an id.  Map all ids back to string name of node.  Construct directed graph while parsing input data.
\item Enumerate all cliques in graph.
\begin{enumerate}
\item This is best done by further processing directed graph obtained in previous part.  A simple and good reduction is to convert directed graph into a directed acyclic graph.  To do this follow this procedure:
\begin{enumerate}
\item Original graph $G = (V,E)$ where $e\in E = (i,j)$ is a tuple describing directed edge from $v_{i} \to v_{j}$ where $v_{i},v_{j}\in V$.
\item We want to obtain $G' = (V',E')$ such that:
\begin{enumerate}
\item $V'=V$.
\item $e'\in E' = (i,j)$ iff $(i,j)\in E$, $(j,i)\in E$ and $j > i$.
\end{enumerate}
\item We can construct a simple algorithm that simply iterates through all edges in $E$ and constructs $G'$ satisfying conditions above.
\end{enumerate}
\item Now that we have $G'$ we will use it to enumerate all possible cliques in $G$ via a simple recursive algorithm:
\begin{enumerate}
\item Function FindCliques.  Input = ($G$, $i$).  $G=(V,E)$ is a DAG subgraph and $i$ is index of vertex that called current incarnation.  Return a list $L$ of all cliques $C$, where $\forall i \in C, v_{i}\in V$.
\item For all every vertex $v_{i}$ in $V$ do:
\begin{enumerate}
\item count = $|e\in (i,j)\in E|$.  (all edges coming out of $v_{i}$.
\item if count == 1 then add and return a list of one 2-clique $(i,j)$ where j is the only child of $v_{i}$).
\item Otherwise add all $(i,j)$ 2-cliques.  $j$ is index of some child of $v_{i}$.
\item Form new subgraph $G'=(V',E')$ s.t. $V'=(v_{j}..), \forall j \in (i,j) \in E$.  $E'=(j,k)$ s.t. $(i,k) \in E$ and $(j,k) \in E$.  In other words we new graph will contain all children of $v_{i}$ and the edges for each child in new graph will be intersection of old edges for child and $v_{i}$'s children.  
\item Set $L' = FindCliques( G', i )$.
\item For every clique $\in L'$ add $i$. 
\item Add L' to L and return L.
\end{enumerate}
\end{enumerate}
\end{enumerate}
\item Remove cliques that are not maximal.  This is pretty easy.  Just look at all pair of cliques and see if any of them is a proper subset of another.  If so remove the one that is smaller.  In code I have optimized this part so it is fast.  If you do it on all cliques returned by FindCliques this would be rather slow.
\item Output maximal cliques as per specification in problem statement.
\end{enumerate}

\section{Analysis}

In my code I have combined step 3 and 4, so that finding maximum cliques is only a bit slower than finding all cliques.  My algorithm is bounded by number of possible cliques in the graph which is $3^{n/3}$ where n is number of vertices.  However it should be polynomial time with respect to number of cliques.  

My code finishes large input given on Blackboard in about 8 seconds on i7 950 @3GHz computer.

\bibliographystyle{acm}
\bibliography{report_ref}
\end{document}
