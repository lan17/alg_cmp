\documentclass[11pt]{article}

\usepackage{epsf,amsfonts,amssymb,hyperref,cite}

\begin{document}
\title{Solution to "It's a small world" facebook puzzle\cite{smallworld}.}
\author{Lev A Neiman}

\maketitle

\section{Problem Statement}

Given a large array of points on a 2-d plane, find closest 3 neighbors for each point.  

\section{Solution}

Solution must be somewhat robust and efficient as simple $ O(Nlog(N)N^{2}) $ (computing distance between all points, then sorting) will both take  up too much memory and run forever.

I have used a simple space partitioning algorithm (also divide and conquer) that splits 2d plane into bins of fixed size.  All bins are disjoint and contain monotonically increasing points with respect to x coordinate.  

I use following data structures:
\begin{enumerate}
\item Each point $i$ is a tuple $P_{i}=\left \{ id,x,y \right \}$.
\item A vector $friends$ which contains tuples $P$ in order given in input file.
\item A vector $xf$ of points sorted by their x coordinate.
\item A vector $yf$ of points sorted by their y coordinate.
\item An associative array $M(id)\to \left \{ (id_{0},d_{0}), (id_{1},d_{1}), (id_{2}, d_{2}) \right \}$, which contains for every id a vector of tuples of a another point's $id_{i}$ and distance $d_{i}$ between id and $id_{i}$.  This vector is kept sorted by increasing d.
\item A set of point ids $redo$ which contains ids of points which need to be re-evaluated after each bin computation.  
\end{enumerate}

Basic operation of algorithm is as follows:
\begin{enumerate}
\item Read all points from file and store as a vector of tuples P.
\item Create vector's $xf$ and $yf$ by sorting points by x and y coordinates respectively.
\item Given a bin size $B$, iteratively divide vector $xf$ into bins each of size $B$.
\item Inside each bin we compute $x_{min}$ and $x_{max}$ which are the minimum and maximum x coordinates inside the bin respectively.
\item For every point inside one bin we compute distances between all other points and store them in $M$.  There is a possible optimization here. If we already computed some possible 3 closest neighbors for a given point we can disregard other points who's difference in x or y coordinates is larger than third smallest distance we currently have.  Thus we do not need to perform an expensive full fledged distance computation.
\item Once we have all possible closest neighbors for a point inside a bin we must check if the distance to third closest neighbor is larger than distance from that point to either $x_{min}$ or $x_{max}$, in which case we put it into $redo$.  Otherwise such point's three current closest neighbors are guaranteed to be globally closest.
\item For each point $P_{r_{i}}$ in $redo$ we must consider points outside the bin as it is quite possible that there are closer points.  However we do not need to consider all points in $friends$.  We can use $yf$ to find all points which are within certain distance in y coordinate of $P_{r_{i}}$.  Points which have difference in y coordinate from $P_{r_{i}}$ larger than $P_{r_{i}}$'s third closest neighbors can be disregarded.

To find such range in $yf$ we can use binary search.  First element would have y coordinate larger or equal to $P_{r_{i}}$'s y coordinate - $P_{r_{i}}$'s third neighbor.  Last element would have y coordinate smaller or equal to $P_{r_{i}}$'s y coordinate + $P_{r_{i}}$'s third neighbor.

An optimization similar to step 5 is possible here, as we can disregard points with absolute coordinate difference larger than current third minimum distance.  Thus we don't have to perform expensive distance computation.
\item Go through $M$ and output answers.
\end{enumerate}

\section{Analysis}

$N$ = number of points in input.  
\begin{itemize}
\item Step 2 must take $O(Nlog(N))$ time as we are doing simple sort two times.  
\item Step 4 is constant time as $xf$ is already sorted and its trivial to extract $x_{min}$ and $x_{max}$ for each bin.  
\item Steps 5 and 6 are $O( \left \lceil \frac{N}{B} \right \rfloor B^2) \approx  O(NB) $, as for each point in a bin we must consider all other points in a bin, and there are $  \left \lceil \frac{N}{B} \right \rfloor $ bins.
\item Step 6's run time depends on size of $redo$ set, which in turn depends on the choice of $B$ and layout of input points.  In worst case $|redo|=N$.  For every point we use binary search inside $yf$ to find first point that is worth considering.  This takes $O(log_{2}(N))$ steps.  In worst case the range inside $yf$ is $N$, therefore worst case complexity is $O(N^2)$.  In practice, however, this range is a lot smaller than $N$ given a close to uniform distribution of points across 2d plane.  We also use optimization whereby we don't have to compute distance between every other point in range for some $P_{r_{i}}$ , but throw out some based on absolute difference in their x and y coordinate from $P_{r_{i}}$.  Therefore the number of actual distance computations is always less than $O(N^2)$, but we may still need to consider that many points in worst case.
\end{itemize}

Run time of the whole algorithm is $O(N^2)$ in worst case, but on average it runs pretty efficiently with a good choice of $B$.  Small $B$ tends to cause the size of $redo$ to grow.  However a very large $B$ dominates the run-time and a subsequent small size of $redo$ doesn't matter.  For the test case given by Dr. Juedes on Blackboard I have determined an optimal value of $B$ to be somewhere around a $[100,1,000]$.  With such $B$ the algorithm finishes the given input in under a second on Intel i7 950 CPU @ 3.05 GHz. and under 2 seconds on Intel Core duo @ 2.20 GHz.  

I suspect there are much more efficient algorithms than this possible as people on Facebook board seem to report much better running times than mine.  There are also a few optimizations possible for my current general approach.  During Step 6 I may re-evaluate points that I have already considered in Step 5.  If I could find a way to limit that, it would probably improve my running time significantly.

However, once the facebook bot is finally back up again I am pretty confident that my solution should pass.

\bibliographystyle{acm}
\bibliography{report_ref}
\end{document}
