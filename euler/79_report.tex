\documentclass[11pt]{article}

\usepackage{epsf,amsfonts,amssymb}

\begin{document}
\title{Solution to Problem 79 in projecteuler.net}
\author{Lev A Neiman}

\maketitle

\section{Problem Statement}

A common security method used for online banking is to ask the user for three random characters from a passcode. For example, if the passcode was 531278, they may ask for the 2nd, 3rd, and 5th characters; the expected reply would be: 317.

The text file, keylog.txt, contains fifty successful login attempts.

Given that the three characters are always asked for in order, analyse the file so as to determine the shortest possible secret passcode of unknown length.

\section{Solution}

I have formed a graph out of the input file.  A node is a digit, and an edge between nodes a and b exists if b follows a.  Given such graph I have determined that there are no loops in the graph, and that the starting digit has to be 7 because no other digit preceeds it in graph.  

Given such constraints it is easy to write a simple DFS to solve the problem:

\begin{verbatim}

def search( i ):
    foll = get_followed(i)
    if len( foll ) == 0:
        return [i]
    ret = []
    for f in foll:
        r = search(f)
        if len(r) > len(ret):
             ret = r
    ret = [i] + ret
    return ret

print solve(7)

\end{verbatim}

\section{Analysis}

Running time is O(N) where N is number of nodes in graph.

\end{document}
